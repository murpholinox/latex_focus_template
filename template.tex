%%%%%%%%%%%%%%%%%%%%%%%%%%%%%%%%%%%%%%%%%
% Focus Beamer Presentation
% LaTeX Template
% Version 1.0 (8/8/18)
%
% This template has been downloaded from:
% http://www.LaTeXTemplates.com
%
% Original author:
% Pasquale Africa (https://github.com/elauksap/focus-beamertheme) with modifications by 
% Vel (vel@LaTeXTemplates.com)
%
% Template license:
% GNU GPL v3.0 License
%
% Important note:
% The bibliography/references need to be compiled with bibtex.
%
%%%%%%%%%%%%%%%%%%%%%%%%%%%%%%%%%%%%%%%%%

%----------------------------------------------------------------------------------------
%	PACKAGES AND OTHER DOCUMENT CONFIGURATIONS
%----------------------------------------------------------------------------------------

\documentclass{beamer}

\usetheme[numbering=fullbar]{focus} % Use the Focus theme supplied with the template
% Add option [numbering=none] to disable the footer progress bar
% Add option [numbering=fullbar] to show the footer progress bar as always full with a slide count

% Uncomment to enable the ice-blue theme
%\definecolor{main}{RGB}{92, 138, 168}
%\definecolor{background}{RGB}{240, 247, 255}

%------------------------------------------------

\usepackage{booktabs} % Required for better table rules
\usepackage[utf8]{inputenc} % Set up input encoding for spanish
\usepackage[T1]{fontenc} % Set up font encoding for spanish
\usepackage[spanish, mexico]{babel} % Translation of environments
\usepackage{dirtytalk} % Allows double quotes with \say{text}
\usepackage{graphicx} % To insert images
\usepackage{siunitx} % Sistema internacional de unidades
\usepackage[version=4]{mhchem} % Ecuaciones químicas.
\graphicspath{ {Images/}}

%----------------------------------------------------------------------------------------
%	 TITLE SLIDE
%----------------------------------------------------------------------------------------

\title{Proyecto X \\ O algo así}

%\subtitle{Subtitle}

\author{Francisco Murphy Pérez \\ Dr. Enrique Rudiño Piñera}

%\titlegraphic{\includegraphics[scale=1.25]{Images/focuslogo.pdf}} % Optional title page image, comment this line to remove it

\institute{Instituto de Biotecnología \\ Universidad Nacional Autónoma de México}

\date{20 04 2020}

%------------------------------------------------

\begin{document}

%------------------------------------------------

\begin{frame}
	\maketitle % Automatically created using the information in the commands above
\end{frame}

%----------------------------------------------------------------------------------------
%	 INTRODUCCIÓN
%----------------------------------------------------------------------------------------

\section{Introducción}

\begin{frame}{Cristalografía de rayos X}

Actualmente la cristalografía de rayos X (CRX), es el principal método con el cual se puede obtener detalle atómico de la proteína de interés. De 163141 estructuras depositadas en la base de datos de proteínas (PDB)\footnote{\url{https://www.rcsb.org/}.}, 145083 se han determinado gracias a este método\footnote{A 25 de abril del 2020.}. Esto representa el 88.93 \% del total.

\end{frame}

\begin{frame}{El experimento de CRX}

En breve, el experimento de CRX consiste en:

\begin{enumerate}
	\item Incidir \alert{rayos X} sobre el cristal de proteína. 
	\item Obtener el patrón de difracción. 
	\item Rotar el cristal en cierto eje. 
	\item Repetir $n$ veces.
\end{enumerate}

\end{frame}

\begin{frame}{Fuentes de rayos X}

La fuente de rayos X más común para realizar un experimento de CRX, es la radiación sincrotrón. De 145083 estructuras determinadas por la CRX, 114781 fueron determinadas en un sincrotrón\footnote{A 25 de abril del 2020}. Esto representa el 79.11 \%. 

\end{frame}

\begin{frame}{Daño por radiación}

Una de las limitantes de la CRX, es el daño por radiación. 

\begin{figure}[h]
    \centering
    \includegraphics[width=0.8\textwidth]{teng.png}
    \caption{Cristal de lisozima y su patrón de difracción después de una dosis de radiación de \SI{120}{\kilo\gray} (izquierda) y \SI{16.7}{\mega\gray} (derecha). Imagen tomada de \cite{Teng2000}.}
    \label{fig:mesh1}
\end{figure}

\end{frame}

\begin{frame}{Daño global}

El daño global se nota en la pérdida de reflexiones y su disminución en intensidad. Después de procesar los datos se nota: en el cambio del volumen de la celda unitaria, en el aumento del factor de escala y/o en las métricas que ayudan a evaluar la calidad de los datos o de la relación entre datos y modelo, pues tienden a empeorar.

\end{frame}

\begin{frame}{Daño específico}
El daño por radiación sobre ciertos residuos de aminoácidos. No ocurre de manera idéntica. Modelo estructural \emph{perturbado}.

\begin{figure}[h]
	\centering
	\includegraphics[width=0.8\textwidth]{prueba3.png}
	\caption{Diferencia de densidad electrónica para un cristal de lisozima entre dos colectas de datos. Estructuras y datos de \cite{Nanao2005}.}
	\label{fig:mesh1}
\end{figure}

\end{frame}

\begin{frame}
	\begin{alertblock}{Nada}
		En el peor de los casos, es imposible obtener un modelo.
	\end{alertblock}
\end{frame}


\begin{frame}{Crioprotección}
Una de las primeras estrategias para minimizar el daño por radiación fue realizar el experimento de difracción a bajas temperaturas.

\begin{figure}[h]
	\centering
	\includegraphics[width=0.8\textwidth]{garman2003.png}
	\caption{Adopción de la crioprotección. Imagen tomada de \cite{Garman2003}.}
	\label{fig:mesh1}
\end{figure}

\end{frame}

\begin{frame}{Radioprotectores}

Interacción con los radicales libres antes que la proteína.

\begin{figure}[h]
	\centering
	\includegraphics[width=0.7\textwidth]{delamora2011.png}
	\caption{Radioprotectores en cristal de lisozima nativo (a), con ácido ascórbico (b) y con nitrato de sodio (c). Imagen tomada de \cite{DeLaMora2011}.}
	\label{fig:mesh1}
\end{figure}

\end{frame}

\begin{frame}{To scavenge or not to scavenge}
A comparación de la crioprotección, el uso de radioprotectores todavía no se ha adoptado como parte de la rutina del experimento de CRX. Al menos dos artículos científicos han debatido sobre este punto \cite{Nowak2009,Allan2013}.    
\end{frame}

%----------------------------------------------------------------------------------------
%	 ANTECEDENTES
%----------------------------------------------------------------------------------------

\section{Antecedentes}

\begin{frame}{pH}

El principal radical libre generado por la radiación, es el electrón solvatado, el cual se encuentra en un equilibrio ácido base.
\begin{equation*}
\ce{e^{-}_{solv.} + H^{+} <=> H^{.}}
\end{equation*}
En una solución ácida el electrón solvatado se convierte en \ce{H^.} y esta especia se recombina generando \ce{H2}, el cual se acumula dentro del cristal macromolecular \cite{Meents2010}. 
En este caso el ion hidronio funciona como radioprotector. 
\end{frame}

\begin{frame}{pH}
En mi tesis de maestría se investigó el efecto del pH en cristales de lisozima: el cristal con el pH más ácido (3.7) presentó mayor daño por radiación que el cristal con el pH más básico (5.7).
\end{frame}

%----------------------------------------------------------------------------------------
%	 MATERIALES Y MÉTODOS
%----------------------------------------------------------------------------------------

\section{Materiales y métodos}

\begin{frame}{Proteínas}

\end{frame}

%----------------------------------------------------------------------------------------
%	 DISCUSIÓN
%----------------------------------------------------------------------------------------

\section{Discusión}

\begin{frame}{Título x}
	Algo de texto por aquí y por allá.
\end{frame}

\begin{frame}{Título x}
	Algo de texto por aquí y por allá.
\end{frame}

\begin{frame}{Título x}
	Algo de texto por aquí y por allá.
\end{frame}

%----------------------------------------------------------------------------------------
%	 CONCLUSIÓN
%----------------------------------------------------------------------------------------

\section{Conclusión}

\begin{frame}{Título x}
	Algo de texto por aquí y por allá.
\end{frame}

%----------------------------------------------------------------------------------------
%	 CLOSING/SUPPLEMENTARY SLIDES
%----------------------------------------------------------------------------------------

\appendix

\begin{frame}[allowframebreaks]{Referencias}
	
	%\nocite{*} % Display all references regardless of if they were cited
	\bibliography{bib.bib}
	\bibliographystyle{unsrt}
\end{frame}

%------------------------------------------------

\begin{frame}{Backup Slide}
	This is a backup slide, useful to include additional materials to answer questions from the audience.
	\vfill
	The package \texttt{appendixnumberbeamer} is used to refrain from numbering appendix slides.
\end{frame}

%----------------------------------------------------------------------------------------

\end{document}
